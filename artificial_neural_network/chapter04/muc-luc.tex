\section{Phụ lục}
Dưới đây là một số thuật ngữ có đề cập trong chương 4:
\clearpage
\begin{table}[htp!]
    \centering
    \caption{Bảng các thuật ngữ}
    \label{tb:thuat_ngu}
    \begin{tabular}{ |p{3cm}|p{8cm}| }
    \hline
    	\textbf{Thuật ngữ} & Ý nghĩa \\
    \hline
    	\textbf{text} \newline \textbf{analyzer} & phân tích văn bản \\
    \hline
    	\textbf{text} \newline \textbf{classification} & phân loại văn bản \\
    \hline
    	\textbf{text} \newline \textbf{summarization} & tóm tắt văn bản \\
    \hline
    	\textbf{corpus} & kho ngữ liệu \\
    \hline
        \textbf{word} & từ–đơn vị mang nghĩa độc lập, được cấu tạo bởi (các) hình vị; có chức năng định danh. \newline Ví dụ: I-am-reading-my–books   \\
    \hline
        \textbf{phrase} & ngữ-gồm hai hay nhiều từ có quan hệ ngữ pháp hay ngữ nghĩa với nhau.\newline Ví dụ: bức thư, mạng máy tính, computer system,...    \\
    \hline
    	\textbf{sentence} & câu-gồm các từ/ngữ có quan hệ ngữ pháp hay ngữ nghĩa với nhau và có chức năng cơ bản là thông báo. \newline Ví dụ: I am reading my books.    \\
    \hline
    	\textbf{text} & văn bản-hệ thống các câu được liên kết với nhau về mặt hình thức, ngữ pháp, ngữ nghĩa và ngữ dụng.   \\
    \hline
    	\textbf{term} & thân từ-có thể bao gồm một hay nhiều hình vị gốc. \newline Ví dụ: babysit   \\
    \hline
    	\textbf{token} & một dãy tuần tự các ký  trong bảng chữ cái, hoặc dãy tuần tự các con số (một chữ số có chứa dấu chấm là dấu chấm thập phân được xem như là một token), hoặc một ký tự không nằm trong bảng chữ cái (như dấu chấm câu, dấu ngoặc kép, các ký tự mở rộng,...)   \\
    \hline
    \end{tabular}
\end{table}



