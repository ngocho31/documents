\chapter{Lời nói đầu}
%\addcontentsline{toc}{chapter}{Acknowledgments}
\setheader{Lời nói đầu}

% Sự ra đời của ``Máy tính'' đã dẫn đến cuộc cách mạng văn minh lần thứ ba, cách mạng thông tin (information revolution), bên cạnh hai cuộc cách mạng nông nghiệp và công nghiệp trước đó. Trước khi có sự ra đời của máy tính, những ứng dụng rất bình thường ngày nay như ứng dụng tìm kiếm thông tin, ứng dụng mạng internet, ứng dụng điện thoại di động,... đều được xem là các ứng dụng viễn tưởng. Với sự ra đời và phát triển mạnh mẽ của các máy tính trong hơn bảy thập kỷ qua, các ứng dụng này không những trở thành sự thật mà còn phổ biến trong đời sống xã hội.

% Sách \emph{``Kiến trúc Máy tính''} này được viết ra với mục đích phục vụ cho các độc giả học tập và nghiên cứu các vấn đề xoay quanh kiến trúc, hoạt động và đánh giá hiệu suất các máy tính. Như vậy có thể thấy rằng, đối tượng khảo sát chính trong sách này là ``máy tính''. Vậy máy tính là gì? Theo định nghĩa của từ điển Cambridge\footnote{phiên bản online tại ``http://dictionary.cambridge.org/''} thì máy tính (chính xác phải gọi là ``máy tính điện tử đa dụng'' - computer) là ``một \textbf{máy điện tử} được sử dụng để lưu trữ, tổ chức và tìm kiếm các từ, số và hình ảnh nhằm mục đích tính toán và điều khiển các máy khác''. Một định nghĩa đơn giản hơn có thể được dùng để định nghĩa máy tính là ``máy tính là một \textbf{thiết bị điện tử đa dụng} có thể được lập trình để thực hiện một tập hợp các tác vụ số học hoặc luận lý một cách \textbf{tự động}''. Điều này có nghĩa rằng, bất kỳ thiết bị nào thỏa mãn định nghĩa trên đều có thể được gọi là máy tính. Vì vậy, dựa theo định nghĩa này, máy tính ngày nay có muôn hình vạn trạng và có nhiều loại khác nhau. Chi tiết về các loại máy tính khác nhau sẽ được trình bày trong sách này.

% Nội dung chính của quyển sách này sẽ xoay quanh chủ đề ``kiến trúc'' của một máy tính, vậy ``kiến trúc máy tính'' là gì? ``kiến trúc máy tính'' được định nghĩa là ``việc lựa chọn và kết nối các thành phần phần cứng một cách khoa học và nghệ thuật nhằm tạo nên các máy tính đáp ứng được yêu cầu về chức năng, hiệu suất và giá thành''\footnote{Định nghĩa tại trang web WWW Computer Architecture Page - \url{http://pages.cs.wisc.edu/~arch/www/}}. Do đó, nội dung chính của quyển sách sẽ xoay quanh các chủ đề \emph{Đánh giá hiệu suất máy tính}, \emph{Kiến trúc tập lệnh}, \emph{Máy tính số học}, \emph{Kiến trúc bộ xử lý} và \emph{Kiến trúc phân cấp bộ xử lý}. Các chủ đề này sẽ được trải dài trong 5 chương. Ở mỗi chương sẽ có các bài tập cũng cố kiến thức sau mỗi chương. Với 5 chương xoay quanh các chủ đề vừa nêu, mục tiêu chính của quyển sách này là sẽ giúp người đọc hiểu được cấu trúc và tổ chức của một hệ thống máy tính cũng như những nguyên tắc hoạt động cơ bản của nó. Ngoài ra, đối với người đọc có kiến thức về các \textbf{ngôn ngữ đặc tả phần cứng}, quyển sách này có thể giúp người đọc thiết kế và hiện thực được các khối chức năng cơ bản của một hệ thống máy tính từ đó xây dựng nên một hệ thống máy tính đơn giản dùng các board mạch khả cấu hình.

% Trong quá trình hoàn thiện quyển sách, tác giả đã nhận được sự giúp đỡ của rất nhiều đồng nghiệp cũng như Ban Chủ nhiệm Khoa Khoa học và Kỹ thuật Máy tính, Trường Đại học Bách Khoa - ĐHQG-HCM. Xin chân thành gửi lời cảm ơn đến các đồng nghiệp trong Khoa và Ban Chủ nhiệm khoa. Tác giả cũng xin chân thành gửi lời cảm ơn đến các tác giả của các tài liệu tham khảo đã cung cấp những thông tin quý báu giúp hoàn thành quyển sách này.

% Trong quá trình biên soạn sách chắc chắn sẽ không thể tránh khỏi những thiếu sót. Tác giả rất mong nhận được sự đóng góp của người đọc. Mọi sự đóng góp xin vui lòng gửi về:\\
% \emph{\bfseries Phạm Quốc Cường}\\
% \emph{Khoa Khoa học và Kỹ thuật Máy tính, Trường Đại học Bách Khoa}\\
% \emph{268 Lý Thường Kiệt, Phường 14, Quận 10, TP.HCM}\\
% \emph{Điện thoại: (08)386487256 - NB: 5843}\\
% \emph{Email: cuongpham@hcmut.edu.vn} 

% \begin{otherlanguage*}{vietnamese}
% \begin{flushright}
% {\makeatletter\itshape
%     \@firstname\ \@lastname \\
%     TP.HCM, tháng 12 năm 2016
% \makeatother}
% \end{flushright}
% \end{otherlanguage*}

