\chapter{Tóm tắt}
%\addcontentsline{toc}{chapter}{Abstract}
\setheader{Tóm tắt}
%\begin{center}
%{\LARGE{Hybrid Interconnect Design for Heterogeneous Hardware Accelerators}}
%\end{center}
%\begin{spacing}{0.95}

%\textbf{Abstract}:
Sách \emph{``Kiến trúc Máy tính''} bao gồm 5 chương mỗi chương gồm hai phần chính là nội dung của chương đó và phần bài tập nhằm củng cố kiến thức. Nội dung chính của các chương như sau:

Chương~\ref{chp:01}: \textbf{Các vấn đề cơ bản trong thiết kế máy tính}. Chương này tập trung chủ yếu vào việc giới thiệu các công nghệ và các thế hệ máy tính từ khi ra đời vào năm 1946 đến nay. Để có thể so sánh được sức mạnh tính toán của các máy tính trong cùng một thế hệ hoặc giữa các thế hệ cần phải so sánh hiệu suất tính toán của chúng. Chương này trình bày cách tính toán đo đạc hiệu suất cũng như công suất tiêu thụ của các hệ thống máy tính.

Chương~\ref{chp:02}: \textbf{Kiến trúc tập lệnh}. Chương này sẽ giới thiệu bốn nguyên tắc thiết kế tập lệnh của một hệ thống máy tính. Tập lệnh MIPS sẽ được lấy làm ví dụ và trình bày chi tiết ba nhóm lệnh là \emph{nhóm lệnh số học và luận lý}, \emph{nhóm lệnh chuyển dữ liệu} và \emph{nhóm lệnh hỗ trợ ra quyết định}. Tuy nhiên, máy tính chỉ có thể hiểu và thực thi được các lệnh máy (là chuỗi các ký số nhị phân). Do đó, chương này cũng trình bày phương pháp mã hóa và các định dạng lệnh máy trong kiến trúc MIPS.

Chương~\ref{chp:03}: \textbf{Bộ tính toán số học}. Chương~\ref{chp:02} trình bày các lệnh tính toán số học trong tập lệnh MIPS với các giá trị số nguyên. Tuy nhiên, các bài toán trong thực tế không chỉ đơn giản cần các số nguyên và các phép toán đơn giản như trên. Do đó, cần phải có những phép tính khác đối với số nguyên như phép tính nhân và chia. Ngoài ra, cần phải có những cách biểu diễn số thực và các phép toán liên quan đến số thực. Mục tiêu chính của Chương~\ref{chp:03} này sẽ trình bày các vấn đề sau: các lệnh nhân và chia trong tập lệnh MIPS chuẩn cũng như kiến trúc phần cứng cho việc tính toán phép nhân và chia. Biểu diễn số thực trong máy tính và các lệnh xử lý số thực trong tập lệnh MIPS cũng sẽ được trình bày trong chương này.

Chương~\ref{chp:04}: \textbf{Bộ xử lý}. Chương này sẽ trình bày hai dạng hiện thực khác nhau của máy tính theo kiến trúc MIPS. Cách hiện thực thứ nhất khá đơn giản khi mà ở đó mỗi lệnh sẽ được hoàn thành trong một chu kỳ xung nhịp. Điều này có nghĩa là chu kỳ xung nhịp phải đủ dài để thời gian thực thi của tất cả các giai đoạn khi thực thi một lệnh bất kỳ phải nhỏ hơn hoặc bằng thời gian một chu kỳ. Mặc dù cách hiện thực này đơn giản nhưng hiệu suất không cao và chỉ mang tính chất tham khảo. Cách hiện thực thứ hai là cách hiện thực được sử dụng nhiều trong thực tế hơn, đó là cách hiện thực theo cơ chế xử lý ống (pipeline). Ở cách hiện thực này, quá trình thực thi một lệnh sẽ được chia thành nhiều giai đoạn khác nhau và mỗi giai đoạn sẽ được hoàn tất trong một chu kỳ. Tuy nhiên, tại một thời điểm sẽ có nhiều giai đoạn của nhiều lệnh khác nhau cùng được thực thi song song. Do đó, hiệu suất của cách hiện thực theo cơ chế xử lý ống sẽ cao hơn so với cách hiện thực theo mô hình đơn giản.

Chương~\ref{chp:05}: \textbf{Hệ thống bộ nhớ phân cấp}. Tổ chức bộ nhớ ảnh hưởng đến hiệu suất của toàn hệ thống bộ nhớ. Do đó, tổ chức bộ nhớ tốt sẽ góp phần tăng đáng kể hiệu suất hệ thống. Bộ nhớ trong một hệ thống máy tính sẽ được thành nhiều lớp với kích thước và tốc độ truy xuất của các lớp khác nhau. Tương ứng với tốc độ truy xuất cao thì công nghệ hiện thực có giá thành cao hơn. Chương này sẽ trình bày các công nghệ hiện thực bộ nhớ khác nhau hiện đang được sử dụng. Dựa vào những công nghệ này, bộ nhớ máy tính sẽ được tổ chức theo mô hình phân cấp với các phương pháp tiếp cận khác nhau nhằm đạt được hiệu suất xử lý cao nhất với giá thành hợp lý nhất.

%\end{spacing}
%\begin{otherlanguage*}{vietnam}
%\begin{flushright}
%{\makeatletter\itshape
%    \@firstname\ \@lastname \\
%    Delft, January 2013
%\makeatother}
%\end{flushright}
%\end{otherlanguage*}

