\chapter{THUẬT TOÁN CÂY QUYẾT ĐỊNH (DECISION TREE)}

Decision Tree có nhiều thuật toán khác nhau.

\begin{itemize}
    \item ID3:\\
    ID3 (Iterative Dichotomiser 3) được phát triển vào nào 1986 bởi Ross Quinlan. Sử dụng lượng thông tin ứng với biến số phân loại sau đó dùng kỹ thuật tham lam (lựa chọn tối ưu địa phương ở mỗi bước đi với hy vọng tìm được tối ưu toàn cục).\\
    Ví dụ như thuật toán tìm đường đi ngắn nhất của Dijkstra.
    \item C4.5:\\
    Được phát triển từ ID3. C4.5 là thuật toán phân lớp dữ liệu dựa trên cây quyết định hiệu quả và phổ biến trong những ứng dụng khai phá cơ sở dữ liệu có kích thước nhỏ.\\
    So với ID3, C4.5 không cần biến số phân loại lượng đặc trưng. Output theo dạng if-then, không hiển thị những phần cành không cần thiết.
    \item C5.0:\\
    Là bản cải tiến của C4.5. Giúp cải thiện vấn đề hiệu năng và sử dụng ít bộ nhớ hơn.
    \item CART:\\
    CART (Classification and Regression Trees) khá giống với C4.5. Được phát triển bởi Breiman năm 1984. Tạo cây phân tích dựa trên biến phân loại, giải thích, mục đính và hồi quy.
\end{itemize}
