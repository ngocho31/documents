\chapter{Hiện thực hệ thống}

\section{Dữ liệu}
\label{sec:database}
Dưới sự hỗ trợ của một cửa hàng thời trang thực tế (Thời trang Hume), đề tài thực hiện việc chuẩn hóa và gán nhãn cho các dữ liệu về sản phẩm, phục vụ cho nhu cầu tư vấn của người dùng. Ngoài ra, đề tài còn thu thập và tổng hợp một số loại dữ liệu khác phục vụ cho các mục đích khác nhau.

\subsection{Dữ liệu sản phẩm}
\label{subsec:productdb}
Dữ liệu sản phẩm này chứa các thông tin sản phẩm dùng để tư vấn, tìm kiếm cho khách hàng một sản phẩm phù hợp với yêu cầu, một bản ghi trong cơ sở dữ liệu tương ứng với một sản phẩm. Mỗi trường thông tin của sản phẩm có thể có các định dạng khác nhau. Cụ thể:

\begin{itemize}
    \item \textbf{name\_product:} tên của sản phẩm, có thể có nhiều alias của tên sản phẩm trong cơ sở dữ liệu sản phẩm, thuộc kiểu dữ liệu chuỗi ký tự (string).
    \item \textbf{size\_product:} kích cỡ của sản phẩm, có thể có nhiều alias của kích cỡ sản phẩm trong cơ sở dữ liệu sản phẩm, thuộc kiểu dữ liệu chuỗi ký tự (string).
    \item \textbf{color\_product:} màu sắc của sản phẩm, có thể có nhiều alias của màu sắc sản phẩm trong cơ sở dữ liệu sản phẩm, thuộc kiểu dữ liệu chuỗi ký tự (string).
    \item \textbf{material\_product:} chất liệu của sản phẩm, có thể có nhiều alias của chất liệu sản phẩm trong cơ sở dữ liệu sản phẩm, thuộc kiểu dữ liệu chuỗi ký tự (string).
    \item \textbf{cost\_product:} đơn giá của sản phẩm, có thể có nhiều alias của đơn giá sản phẩm trong cơ sở dữ liệu sản phẩm, thuộc kiểu dữ liệu chuỗi ký tự (string).
    \item \textbf{amount\_product:} số lượng sản phẩm hiện tại, có thể có nhiều alias của số lượng sản phẩm trong cơ sở dữ liệu sản phẩm, thuộc kiểu dữ liệu số (int).
\end{itemize}

Bản ghi mẫu của dữ liệu sản phẩm như ví dụ \ref{exam:productrecord}.

\renewcommand{\lstlistingname}{Ví dụ}
\begin{lstlisting}[caption={Một bản ghi của dữ liệu sản phẩm},label={exam:productrecord},language=code_vn,firstnumber=1]
{
    "name_product": "set trắng cổ nơ",
    "size_product": "M",
    "color_product": "None",
    "cost_product": "260",
    "material_product": "vải cao cấp, mặc mát",
    "amount_product": 10
}
\end{lstlisting}

\subsection{Dữ liệu kích cỡ}
\label{subsec:sizedb}
Dữ liệu kích cỡ này chứa các thông tin về kích thước dùng để tư vấn, tìm kiếm cho khách hàng một kích cỡ sản phẩm phù hợp với số đo cơ thể, một bản ghi trong cơ sở dữ liệu tương ứng với một bảng kích thước. Mỗi trường thông tin của dữ liệu có thể có các định dạng khác nhau. Các giá trị của cùng một thông tin có cùng một đơn vị. Cụ thể:

\begin{itemize}
    \item \textbf{size\_customer:} kích cỡ sản phẩm phù hợp với khách hàng, có thể có nhiều alias của kích cỡ sản phẩm trong cơ sở dữ liệu kích cỡ, thuộc kiểu dữ liệu chuỗi ký tự (string).
    \item \textbf{waist\_customer:} số đo vòng eo của cơ thể, có thể có nhiều alias của số đo vòng eo trong cơ sở dữ liệu kích cỡ, thuộc kiểu dữ liệu chuỗi ký tự (string), đơn vị là centimet (cm).
    \item \textbf{height\_customer:} số đo chiều cao của cơ thể, có thể có nhiều alias của số đo chiều cao trong cơ sở dữ liệu kích cỡ, thuộc kiểu dữ liệu chuỗi ký tự (string), đơn vị là centimet (cm).
    \item \textbf{weight\_customer:} số đo cân nặng của cơ thể, có thể có nhiều alias của số đo cân nặng trong cơ sở dữ liệu kích cỡ, thuộc kiểu dữ liệu chuỗi ký tự (string), đơn vị là kilogram (kg).
\end{itemize}

Bản ghi mẫu của dữ liệu kích cỡ như ví dụ \ref{exam:sizerecord}.

\renewcommand{\lstlistingname}{Ví dụ}
\begin{lstlisting}[caption={Một bản ghi của dữ liệu kích cỡ},label={exam:sizerecord},language=code_vn,firstnumber=1]
{
    "size_customer": "S",
    "waist_customer": "65",
    "height_customer": "164",
    "weight_customer": "50"
}
\end{lstlisting}

\subsection{Từ điển thông tin sản phẩm}
Từ điển (Dictionary) của thông tin sản phẩm chứa tất cả các giá trị có thể có của từng thông tin sản phẩm (như mô tả ở mục \ref{subsec:productdb}). Dữ liệu được sử dụng để tạo ra các \textit{User Goal} (mục tiêu người dùng), phục vụ cho việc huấn luyện. Ngoài ra được sử dụng bởi \textit{EMC} (bộ giả lập lỗi), bộ tạo ra các lỗi ngẫu nhiên của người dùng. Mỗi trường thông tin có định dạng danh sách (array). Được biểu diễn rút gọn như ví dụ \ref{exam:dictproduct}.

\renewcommand{\lstlistingname}{Ví dụ}
\begin{lstlisting}[caption={Từ điển thông tin sản phẩm},label={exam:dictproduct},language=code_vn,firstnumber=1]
{
    "name_product": ["đầm nude cổ vest dáng dài", "đầm sơ mi carô", "set hồng vest váy ngắn", "đầm nude lưới tay phồng", ...],
    "size_product": ["S", "M", "L", ...],
    "color_product": ["nude", "carô xanh", "hồng", "xanh", ...],
    "cost_product": ["280", "210", "260", "238", "195", "350", ...],
    "material_product": ["vải cao cấp, mặc mát", ...],
    "amount_product": ["2", "5", "hai", "9", "ba", "4", "3", "6", ...]
}
\end{lstlisting}

\subsection{Từ điển dữ liệu kích cỡ}
Từ điển (Dictionary) của bảng dữ liệu kích cỡ chứa tất cả các giá trị có thể có của từng trường thông tin trong dữ liệu kích cỡ (như mô tả ở mục \ref{subsec:sizedb}). Dữ liệu được sử dụng để tạo ra các \textit{User Goal} (mục tiêu người dùng), phục vụ cho việc huấn luyện. Ngoài ra được sử dụng bởi \textit{EMC} (bộ giả lập lỗi), bộ tạo ra các lỗi ngẫu nhiên của người dùng. Mỗi trường thông tin có định dạng danh sách (array). Được biểu diễn rút gọn như ví dụ \ref{exam:dictsize}.

\renewcommand{\lstlistingname}{Ví dụ}
\begin{lstlisting}[caption={Từ điển dữ liệu kích cỡ},label={exam:dictsize},language=code_vn,firstnumber=1]
{
    "size_customer": ["L", "S", "M"],
    "waist_customer": ["anything", "66", "73", "68", "63", "58", ...],
    "height_customer": ["163", "159", "166", "150", "anything", "155", "165", "156", ...],
    "weight_customer": ["56", "45", "47", "46", "57", "48", "41", "52", "anything", ...]
}
\end{lstlisting}

\subsection{Mục tiêu người dùng (User Goal)}
\label{subsec:usergoal}
Mục tiêu người dùng (User Goal) chứa danh sách các trường hợp biểu diễn mục tiêu có thể có của người dùng. Dữ liệu được sử dụng bởi bộ mô phỏng người dùng sử dụng như mục tiêu thật của người dùng, phục vụ cho quá trình huấn luyện. Mỗi \textit{goal} chỉ chứa cho một mục tiêu xuyên suốt cuộc hội thoại. Cấu trúc cho một \textit{goal} như sau:

\begin{itemize}
    \item \textbf{intent:} mục tiêu chính của người dùng trong cuộc hội thoại, kiểu dữ liệu chuỗi ký tự (string). Các mục tiêu khi tham gia hội thoại của người dùng có thể có là:
    \begin{itemize}
        \item \textbf{request:} yêu cầu thông tin sản phẩm, hoặc tư vấn kích cỡ sản phẩm cho khách hàng
        \item \textbf{order:} mục tiêu đặt hàng sản phẩm của khách hàng
    \end{itemize}
    \item \textbf{inform\_slots:} chứa tất cả các thông tin cần thông báo cho tác nhân trong suốt cuộc hội thoại. Các thông tin cần thông báo có cấu trúc <tên thông tin>: <giá trị của thông tin>. Trong đó:
    \begin{itemize}
        \item \textbf{<tên thông tin>:} tên thông tin là một trong các trường thông tin đã được định nghĩa ở mục \ref{subsec:productdb} và \ref{subsec:sizedb}.
        \item \textbf{<giá trị của thông tin>:} giá trị của thông tin này có thể nằm trong bộ từ điển của nó hoặc không.
    \end{itemize}
    \item \textbf{request\_slots:} chứa tất cả các thông tin yêu cầu tác nhân cung cấp cho đến khi kết thúc cuộc hội thoại. Các thông tin yêu cầu có cấu trúc <tên thông tin>: "UNK". Trong đó:
    \begin{itemize}
        \item \textbf{<tên thông tin>:} tên thông tin là một trong các trường thông tin đã được định nghĩa ở mục \ref{subsec:productdb} và \ref{subsec:sizedb}.
        \item \textbf{"UNK":} giá trị mặc định, có nghĩa "chưa biết", cần được thông báo/ điền vào từ tác nhân.
    \end{itemize}
\end{itemize}

Bản ghi mẫu của mục tiêu người dùng như ví dụ \ref{exam:goalrecord}.

\renewcommand{\lstlistingname}{Ví dụ}
\begin{lstlisting}[caption={Một bản ghi của mục tiêu người dùng},label={exam:goalrecord},language=code_vn,firstnumber=1]
{
    "intent": "request",
    "inform_slots": {
        "name_product": "set dây kèm quần",
        "amount_product": "4",
        "color_product": "đen viền",
        "waist_customer": "63",
        "height_customer": "158",
        "weight_customer": "59"
    },
    "request_slots": {
        "size_product": "UNK",
        "material_product": "UNK",
        "cost_product": "UNK"
    }
}
\end{lstlisting}

\subsection{Hội thoại (Dialog)}
Dữ liệu hội thoại (Dialog) chứa danh sách các cuộc hội thoại tư vấn, như mô tả ở mục \ref{subsubsec:dialog}. Dữ liệu được sử dụng cho giai đoạn khởi động. Mỗi \textit{dialog} mô tả quá trình hội thoại tư vấn diễn ra như trong thực tế, chứa danh sách các hành động lần lượt là của người dùng và tác nhân, mỗi hành động có cấu trúc như sau:

\begin{itemize}
    \item \textbf{speaker:} mô tả lượt hành động do người dùng hay tác nhân thực hiện, kiểu dữ liệu là chuỗi ký tự (string).
    \item \textbf{intent:} ý định hành động của người dùng hoặc tác nhân trong lượt thoại hiện tại, kiểu dữ liệu chuỗi ký tự (string). Các mục tiêu có thể có được mô tả ở bảng \ref{tab:userintent}.
    \item \textbf{inform\_slots:} chứa các thông tin cần thông báo trong lượt thoại hiện tại. Các thông tin cần thông báo có cấu trúc <tên thông tin>: <giá trị của thông tin>. Trong đó:
    \begin{itemize}
        \item \textbf{<tên thông tin>:} tên thông tin là một trong các trường thông tin đã được định nghĩa ở mục \ref{subsec:productdb} và \ref{subsec:sizedb}.
        \item \textbf{<giá trị của thông tin>:} giá trị của thông tin này có thể nằm trong bộ từ điển của nó hoặc không.
    \end{itemize}
    \item \textbf{request\_slots:} chứa các thông tin yêu cầu đối phương cung cấp trong lượt thoại hiện tại. Trường thông tin yêu cầu này là danh sách các thông tin có kiểu dữ liệu là chuỗi ký tự (string).
    \begin{itemize}
        \item \textbf{<tên thông tin>:} tên thông tin là một trong các trường thông tin đã được định nghĩa ở mục \ref{subsec:productdb} và \ref{subsec:sizedb}.
    \end{itemize}
\end{itemize}

Bản ghi mẫu của hội thoại như ví dụ \ref{exam:dialogrecord}.

\renewcommand{\lstlistingname}{Ví dụ}
\begin{lstlisting}[caption={Một đoạn hội thoại},label={exam:dialogrecord},language=code_vn,firstnumber=1]
[
    {"speaker": "user", "intent": "request", "inform_slots": {}, "request_slots": ["cost_product"]},
    {"speaker": "agent", "intent": "request", "inform_slots": {}, "request_slots": ["name_product"]},
    {"speaker": "user", "intent": "inform", "inform_slots": {"name_product": "đầm body vest"}, "request_slots": []},
    {"speaker": "agent", "intent": "inform", "inform_slots": {"cost_product": "260"}, "request_slots": []},
    {"speaker": "user", "intent": "ok", "inform_slots": {}, "request_slots": []},
    {"speaker": "agent", "intent": "match_found", "inform_slots": {}, "request_slots": []},
    {"speaker": "user", "intent": "ok", "inform_slots": {}, "request_slots": []},
    {"speaker": "agent", "intent": "done", "inform_slots": {}, "request_slots": []},
    {"speaker": "user", "intent": "done", "inform_slots": {}, "request_slots": []}
]
\end{lstlisting}

\section{Bộ xử lý phản hồi người dùng}
Do giới hạn của đề tài, trong luận án này không xử lý ngôn ngữ tự nhiên cho các câu phản hồi của người dùng. Tuy nhiên, hệ thống có thể tích hợp với bộ xử lý ngôn ngữ tự nhiên khác. Trong bộ xử lý phản hồi người dùng này, hệ thống sẽ tiến hành xử lý các thông tin nhập vào từ giao diện hệ thống Chabot hoặc kết quả trả về từ bộ xử lý ngôn ngữ tự nhiên bên ngoài thành một cấu trúc đồng nhất để sử dụng trong lõi hệ thống, lõi hệ thống như mô tả ở hình \ref{fig:chatbotapp}. Các hành động được biểu diễn như cấu trúc \ref{struct:action}.

\renewcommand{\lstlistingname}{Cấu trúc}
\begin{lstlisting}[caption={Cấu trúc cho một hành động},label={struct:action},language=structure,firstnumber=1]
{
    "intent": <value 1>,
    "inform_slots": <value 2>,
    "request_slots": <value 3>
}
\end{lstlisting}

Trong đó:

\begin{itemize}
    \item \textbf{<value 1>}: chứa ý định người dùng, các giá trị hợp lệ được mô tả trong bảng \ref{tab:userintent}
    \item \textbf{<value 2>}: là một dictionary (kiểu dữ liệu của Python), có cấu trúc key:value, với key là tên thông tin cung cấp, các trường thông tin được định nghĩa ở mục \ref{subsec:productdb} và \ref{subsec:sizedb}, value là giá trị của thông tin
    \item \textbf{<value 3>}: là một dictionary, có cấu trúc key:value, với key là tên thông tin yêu cầu, các trường thông tin được định nghĩa ở mục \ref{subsec:productdb} và \ref{subsec:sizedb}, value có giá trị mặc định là chuỗi kí tự "UNK".
\end{itemize}

\begin{table}[!ht]
\caption{Các ý định hành động của người dùng}
\centering
\begin{tabular}{|p{1cm}|c|p{11cm}|}
\hline
\centering\textbf{STT} & 
\centering\textbf{Ý định} & 
\parbox[t]{11cm}{\centering\textbf{Mô tả}} \\ % inserts table %heading
\hline
\centering 1 & 
hello & 
Ý định khi người dùng thực hiện các câu chào, hỏi thăm với tác nhân \\
\hline
\centering 2 & 
inform & 
Ý định khi người dùng cần thông báo thông tin đến cho tác nhân \\
\hline
\centering 3 & 
request & 
Ý định khi người dùng yêu cầu tác nhân cung cấp thông tin \\
\hline
\centering 4 & 
order & 
Ý định khi người dùng muốn đặt hàng sản phẩm \\
\hline
\centering 5 & 
ok & 
Ý định khi người dùng thể hiện ý đồng tình, chỉ cho tác nhân biết rằng nó đã làm điều gì đó tốt, đúng hoặc người dùng đã sẵn sàng kết thúc cuộc trò chuyện \\
\hline
\centering 6 & 
reject & 
Ý định khi người dùng thể hiện ý phản đối, không đồng tình, với thông tin tác nhân cung cấp \\
\hline
\centering 7 & 
done & 
Ý định khi người dùng xác nhận kết thúc hội thoại. Đối với bộ mô phỏng người dùng, diễn ra khi nó nhận thấy tác nhân hoàn thành được yêu cầu hoặc khi hội thoại diễn ra quá lâu. \\
\hline
\centering 8 & 
other & 
Các ý định khác của người dùng chưa được định nghĩa trong hệ thống. \\
\hline
\end{tabular}
\label{tab:userintent}
\end{table}

Một hành động người dùng được trình bày ở ví dụ \ref{exam:action}. Tương ứng một câu thoại của người dùng ở thực tế là:

\textit{Cho mình hỏi áo hoa có những màu gì vậy shop?}

\renewcommand{\lstlistingname}{Ví dụ}
\begin{lstlisting}[caption={Ví dụ cho một hành động},label={exam:action},language=code_vn,firstnumber=1]
{
    "intent": "request",
    "inform_slots": {"name_product": "áo hoa"},
    "request_slots": {"color_product": "UNK"}
}
\end{lstlisting}

Các công việc trên sẽ được thực hiện qua hàm $process\_user\_response$, với giao thức được mô tả ở đoạn mã \ref{func:processuser}.

\renewcommand{\lstlistingname}{Hàm}
\begin{lstlisting}[caption={Hàm xử lý phản hồi người dùng},label={func:processuser},language=python,firstnumber=1]
def process_user_response(user_action):
    """
    Processing user response.
    Parameters:
        user_action (dict or string): The user input
    Returns:
        dict: The user action has valid format.
    """
\end{lstlisting}

Hàm này thực hiện các công việc sau:

\begin{itemize}
    \item Kiểm tra đầu vào của người dùng là ở dạng chuỗi hay dạng dictionary.
    \item Nếu đầu vào là ở dạng chuỗi, tiếp tục xử lý theo các bước sau:
    \begin{itemize}
        \item Gửi nó sang một Bộ xử lý ngôn ngữ tự nhiên được xây dựng ở một hệ thống khác (nếu có) với phương thức POST. Kết quả trả về là gồm ý định người dùng, và các thông tin thông báo, yêu cầu.
        \item Chuyển đổi cấu trúc đầu ra của hệ thống này thành cấu trúc hành động như mô tả ở cấu trúc \ref{struct:action}
    \end{itemize}
    \item Hành động đầu vào hoặc đã xử lý ở dạng dictionary,  sẽ được chuẩn hóa một số thông tin tùy vào ý định hiện tại.
    \begin{itemize}
        \item Nếu ý định của người dùng không thuộc các ý định hợp lệ được mô tả ở bảng \ref{tab:userintent}. Ta gán lại "intent" thành một giá trị duy nhất là "other".
        \item Kiểm tra nếu có bất kỳ thông tin nào có giá trị được thông báo (khác giá trị mặc định "UNK"), thì gán cặp tên thông tin và giá trị của nó vào "inform\_slots".
        \item Kiểm tra nếu ý định người dùng là "inform", và có thông tin được yêu cầu thông báo (giá trị mặc định "UNK"), gán lại giá trị "intent" là "request".
        \item Chuẩn hóa một số thông tin nhập vào bởi người dùng về cùng một đơn vị giống với giá trị lưu trữ ở cơ sở dữ liệu (mô tả ở mục \ref{subsec:productdb} và \ref{subsec:sizedb}) để thuận lợi cho việc truy xuất dữ liệu được chính xác. 
    \end{itemize}
\end{itemize}

\section{Bộ quản lý trạng thái hội thoại}
\label{sec:statetracker}
Công việc chính của bộ quản lý trạng thái hội thoại là chuẩn bị trạng thái (state) cho tác nhân. Như đã đề cập ở mục \ref{sec:model}, cần một trạng thái hữu ích để có thể làm đầu vào mạng Deep Q-Learning và lấy ra hành động một cách chính xác. Bộ quản lý trạng thái hội thoại cập nhật trạng thái lịch sử của hội thoại bằng việc thu thập toàn bộ hành động của người dùng và cả tác nhân từ lúc bắt đầu hội thoại cho tới thời điểm hiện tại. Đồng thời nó cũng theo dõi tất cả các thông tin đã được thông báo có trong bất kỳ hành động của người dùng hay của tác nhân trong hội thoại hiện tại. Trạng thái được sử dụng bởi tác nhân được biểu diễn dưới dạng mảng và mã hóa các thông tin như hình \ref{exam:state}. Bên cạnh đó, khi tác nhân muốn thông báo một giá trị của thông tin cho người dùng, bộ quản lý trạng thái hội thoại sẽ thực hiện công việc truy vấn lên cơ sở dữ liệu để lấy thông tin đó.

\renewcommand{\lstlistingname}{Ví dụ}
\begin{lstlisting}[caption={Ví dụ một trạng thái hội thoại},label={exam:state},language=exam_en,firstnumber=1]
[1.   0.   0.   0.   0.   0.   0.   0.   0.   0.   0.   0.   1.   0.
 0.   0.   0.   0.   0.   0.   0.   0.   0.   0.   0.   0.   0.   1.
 0.   0.   0.   0.   0.   0.   0.   0.   0.   0.   0.   0.   0.   0.
 0.   0.   0.   0.   0.   0.   1.   0.   0.   0.   1.   0.   0.   0.
 0.   0.   1.   0.   0.   1.   0.6  0.   0.   1.   0.   0.   0.   0.
 0.   0.   0.   0.   0.   0.   0.   0.   0.   0.   0.   0.   0.   1.
 1.   1.   1.   1.   1.   1.   1.   1.   0.   1.   0.04 0.04 0.04 0.04
 0.04 0.04 6.48 0.04 0.04 0.   0.04]
\end{lstlisting}

Một số thông tin được mã hóa (chủ yếu ở dạng one hot encoding), trạng thái bao gồm:

\begin{itemize}
    \item Ý định của người dùng trong hành động gần nhất
    \item Thông tin được người dùng thông báo trong hành động gần nhất
    \item Thông tin mà người dùng yêu cầu trong hành động gần nhất
    \item Ý định của tác nhân trong hành động gần nhất
    \item Thông tin được tác nhân thông báo trong hành động gần nhất
    \item Thông tin mà tác nhân yêu cầu trong hành động gần nhất
    \item Tất cả các thông tin được người dùng thông báo từ lúc bắt đầu cuộc hội thoại cho tới thời điểm hiện tại
    \item Số lượt qua lại diễn ra trong hội thoại
    \item Kết quả trả về sau khi truy vấn lên cơ sở dữ liệu
\end{itemize}

Sau khi tổng hợp toàn bộ các thông tin trên và nối lại thành một mảng hoàn chỉnh, ta được một biểu diễn trạng thái với kích thước 105. Đồng thời, khi hành động được gửi tới cho bộ quản lý trạng thái hội thoại từ tác nhân là inform (cung cấp thông tin yêu cầu) hoặc match\_found (cung cấp toàn bộ thông tin sản phẩm) thì nó sẽ gọi lệnh truy vấn lên cơ sở dữ liệu để lấy thông tin về điền vào hành động.

\section{Bộ sinh phản hồi}
\label{sec:agentresponse}
\subsection{Kiến trúc tổng quát và quá trình huấn luyện mô hình}
Để đưa ra quyết định phản hồi hành động cho người dùng, cần có một hệ thống huấn luyện một mô hình phù hợp. Kiến trúc của hệ thống và quá trình huấn luyện được mô tả ở mục \ref{sec:trainingmodel}. Phần này mô tả cụ thể chi tiết hiện thực mạng nơ-ron.

\subsection{Tập dữ liệu}
Sử dụng toàn bộ dữ liệu đã được mô tả ở mục \ref{sec:database} cho việc huấn luyện mô hình. Các dữ liệu được lưu trữ ở dạng tệp (file) để tiện cho việc truy vấn khi huấn luyện. Cụ thể số liệu các dữ liệu được sử dụng như sau:

\begin{itemize}
    \item File dữ liệu thông tin sản phẩm chứa thông tin về 100 sản phẩm khác nhau.
    \item File dữ liệu về kích cỡ chứa thông tin về 1000 bảng kích cỡ khác nhau.
    \item File từ điển thông tin sản phẩm cũng được sinh ra với 60 giá trị khác nhau của 6 loại thông tin sản phẩm.
    \item File từ điển kích cỡ cũng được sinh ra với 90 giá trị khác nhau của 4 loại thông tin kích cỡ.
    \item File mục tiêu người dùng chứa 100 mục tiêu được sinh ngẫu nhiên về số lượng các thông tin bên trong để mô phỏng sự đa dạng về yêu cầu của người dùng.
    \item File hội thoại chứa 2000 hội thoại được tạo từ bộ luật mô tả đầy đủ các tình huống yêu cầu của người dùng.
\end{itemize}

\subsection{Mạng Deep Q-Learning}
\label{subsec:agent}
Sau nhiều lần xây dựng và thử nghiệm, đề tài hiện thực mô hình huấn luyện với các thông số như sau:

\begin{itemize}
    \item Tầng đầu vào: là trạng thái hiện tại của hội thoại, có kích thước 105, cụ thể được đề cập ở mục \ref{sec:statetracker}
    \item Tầng ẩn: mạng chỉ có một tầng ẩn, và có kích thước là 80.
    \item Tầng đầu ra: là các hành động có thể có của tác nhân, có kích thước 16, là tổ hợp của các ý định tác nhân cùng với các thông tin sản phẩm và kích cỡ, các ý định tác nhân được mô tả cụ thể ở bảng \ref{tab:agentintent}
    \item Kích thước một bó (batch size): 16
    \item Learning rate ($\alpha$): 1e-3
    \item Gamma ($\gamma$): 0.9
\end{itemize}

\begin{table}[!ht]
\caption{Các ý định hành động của tác nhân}
\centering
\begin{tabular}{|p{1cm}|c|p{10cm}|}
\hline
\centering\textbf{STT} & 
\centering\textbf{Ý định} & 
\parbox[t]{11cm}{\centering\textbf{Mô tả}} \\ % inserts table %heading
\hline
\centering 1 & 
hello & 
Ý định khi tác nhân thực hiện các câu chào, hỏi thăm với người dùng \\
\hline
\centering 2 & 
inform & 
Ý định khi tác nhân cung cấp giá trị thông tin đến cho người dùng \\
\hline
\centering 3 & 
request & 
Ý định khi tác nhân muốn người dùng cung cấp giá trị thông tin cần thiết cho hoạt động tư vấn hoặc để chốt đơn hàng \\
\hline
\centering 4 & 
done & 
Ý định khi tác nhân muốn xác nhận kết thúc hội thoại. \\
\hline
\centering 5 & 
match\_found & 
Ý định khi tác nhân muốn cho người dùng biết rằng nó có một kết quả phù hợp mà nó cho rằng sẽ hoàn thành mục tiêu của người dùng \\
\hline
\end{tabular}
\label{tab:agentintent}
\end{table}

\section{Bộ hỗ trợ truy vấn dữ liệu}
Công việc chính của bộ này là hỗ trợ truy vấn dữ liệu cho bộ quản lý trạng thái hội thoại (mô tả ở mục \ref{sec:statetracker}. Bộ hỗ trợ truy vấn dữ liệu nhận vào điều kiện được lập ra từ các thông tin cung cấp từ người dùng và trả về kết quả là giá trị gợi ý, thông tin của sản phẩm hoặc các kết quả hiện tại của quá trình tìm kiếm, tùy thuộc vào trạng thái của hội thoại.

\subsubsection{Hàm đếm số lượng giá trị cho các thông tin}

\renewcommand{\lstlistingname}{Hàm}
\begin{lstlisting}[caption={Hàm đếm số lượng giá trị cho các thông tin},language=python,firstnumber=1]
def get_db_results_for_slots(self, current_informs):
    """
    Parameters:
        current_inform (dict): The current informs/constraints

    Returns:
        dict: Each key in current_informs with the count of the number of matches for that key
    """
\end{lstlisting}

Hàm này có nhiệm vụ nhận vào điều kiện ở dạng \textit{key:value}, và trả về số lượng kết quả thỏa mãn cho từng điều kiện đơn cũng như toàn bộ điều kiện. Cụ thể:

\begin{itemize}
    \item Đếm số lần xuất hiện của mỗi vị trí thông báo hiện tại (key và value) trong các bản ghi cơ sở dữ liệu.
    \item Đối với mỗi bản ghi trong cơ sở dữ liệu và mỗi vị trí thông báo hiện tại nếu vị trí đó nằm trong bản ghi cơ sở dữ liệu (khớp với key và value) thì tăng số lượng cho key đó lên 1.
    \item Nếu không có bất kỳ thông tin được thông báo nào, hãy trả lại tất cả 0.
\end{itemize}

\subsubsection{Hàm đếm giá trị cho một thông tin}

\renewcommand{\lstlistingname}{Hàm}
\begin{lstlisting}[caption={Hàm đếm giá trị cho một thông tin},language=python,firstnumber=1]
def _count_slot_values(self, key, db_subdict):
    """
    Parameters:
        key (string): The key to be counted
        db_subdict (dict): A sub-dict of the database

    Returns:
        dict: The values and their occurrences given the key
    """
\end{lstlisting}

Hàm này có nhiệm vụ nhận vào loại thông tin cần đếm (ở dạng chuỗi) và danh sách các sản phẩm (ở dạng dictionary) và trả về số lần xuất hiện của từng giá trị (thuộc thông tin cần đếm) trong danh sách các sản phẩm.

\subsubsection{Hàm lấy thông tin sản phẩm}

\renewcommand{\lstlistingname}{Hàm}
\begin{lstlisting}[caption={Hàm lấy thông tin sản phẩm},language=python,firstnumber=1]
def get_db_results(self, constraints):
    """
    Parameters:
        constraints (dict): The current informs

    Returns:
        array dict: The available items in the database
    """
\end{lstlisting}

Hàm này có nhiệm vụ nhận vào điều kiện ở dạng \textit{key:value}. Nó xem xét từng bản ghi trong cơ sở dữ liệu và nếu các thông tin của nó chứa tất cả các ràng buộc và giá trị của chúng khớp nhau thì bản ghi đó sẽ được thêm vào danh sách trả về.

\subsubsection{Hàm điền giá trị cho thông tin}

\renewcommand{\lstlistingname}{Hàm}
\begin{lstlisting}[caption={Hàm điền giá trị cho thông tin},language=python,firstnumber=1]
def fill_inform_slot(self, inform_slot_to_fill, current_inform_slots, entity_list):
    """
    Parameters:
        inform_slot_to_fill (dict): Inform slots to fill with values
        current_inform_slots (dict): Current inform slots with values from the StateTracker

    Returns:
        dict: inform_slot_to_fill filled with values
    """
\end{lstlisting}

Hàm này có nhiệm vụ nhận vào thông tin cần điền, lấy các sản phẩm thỏa điều kiện (dựa vào hàm $get\_db\_results$), sau đó đếm giá trị trong thông tin cần điền ($\_count\_slot\_values$) và lấy ra thông tin có số lần xuất hiện nhiều nhất điền vào thông tin.

\section{Bộ sinh câu phản hồi}
Ở mục \ref{sec:agentresponse} đã đề cập đến việc cách mà hệ thống phản hồi lại người dùng thông qua \textit{action} (hành động) - là một dạng biểu diễn thông tin mà mô hình có thể hiểu và sử dụng để giao tiếp giữa các thành phần với nhau và huấn luyện. Tuy nhiên, không thể sử dụng trực tiếp chúng để giao tiếp với người dùng. Để người dùng có thể hiểu được và giao tiếp một cách dễ dàng trong suốt quá trình diễn ra hội thoại, ta cần biểu diễn hành động dưới dạng câu thoại sử dụng ngôn ngữ tự nhiên, sử dụng theo đúng cấu trúc ngữ pháp Tiếng Việt. Để làm được điều này, sử dụng các mẫu câu đã được định nghĩa sẵn và để trống các trường thông tin và được điền vào sau khi truy vấn vào cơ sở dữ liệu. Cụ thể, chia các mẫu câu phản hồi vào từng các nhóm nhỏ dựa vào \textit{intent}, mô tả ở bảng \ref{tab:agentintent}.

Các câu hội thoại mẫu được lưu trữ với cấu trúc như ví dụ \ref{exam:actionnl}.

\renewcommand{\lstlistingname}{Ví dụ}
\begin{lstlisting}[caption={Câu hội thoại mẫu},label={exam:actionnl},language=code_vn,firstnumber=1]
"hello": [
    "text": "Cảm ơn bạn đã quan tâm đến shop, bạn cần shop tư vấn sản phẩm nào ạ",
    "text": "Chào bạn ạ, bạn cần shop hỗ trợ gì ạ", ...
],
"request": [
    {
        "request_slots": [
            "waist_customer"
        ],
        "text": "Cho em xin số đo vòng eo của bạn, e tư vấn cho mình size phù hợp ạ."
    },
    {
        "request_slots": [
            "color_product"
        ],
        "inform_slots": [],
        "text": "dạ bạn muốn lấy màu gì ạ?"
    }, ...
],
"inform": [
    {
        "inform_slots": [
            "size_customer"
        ],
        "text": "Bạn mặc size *size_customer* là chuẩn xinh nha"
    },
    {
        "inform_slots": [
            "name_product"
        ],
        "text": "dạ bên em có nhiều sản phẩm đẹp và chất lượng lắm ạ. Chị tham khảo các mẫu sau *name_product*"
    },
],
...
\end{lstlisting}

\section{Bộ mô phỏng người dùng}
\label{sec:usersim}
Mục đích của bộ mô phỏng người dùng là dùng để mô phỏng người dùng thật để tương tác với tác nhân cũng như chấm điểm nó. Việc này sẽ tốn rất nhiều thời gian nếu như làm với người thật. Bộ mô phỏng người dùng này sẽ được hiện thực theo dạng luật định sẵn. Cụ thể với mỗi lần tham gia cuộc hội thoại, ta đều quy định sẵn các mục tiêu người dùng (\textit{user goal}) và thực hiện các hành động phù hợp với mục tiêu đó, mỗi hành động sẽ kèm theo các thông tin mà nó sẽ yêu cầu hay thông báo.

Mục tiêu người dùng có thể có được từ các cuộc hội thoại thực hoặc được làm thủ công (hoặc cả hai). Mỗi mục tiêu người dùng gồm các \textit{inform slots} (các cặp tên thông tin và giá trị cần được thông báo) và \textit{request slots} (các thông tin yêu cầu cung cấp bởi tác nhân). Ví dụ, ta có một cuộc hội thoại sau đây:

\renewcommand{\lstlistingname}{Ví dụ}
\begin{lstlisting}[caption={Một cuộc hội thoại tư vấn},label={exam:examdialog},language=exam_vn,firstnumber=1]
User: Mẫu áo AT001 này có size XL không bạn?
Admin: Có bạn nha. 
User: Có những màu nào nữa vậy bạn.
Admin: Mẫu áo AT001 này có 3 màu ạ. Đỏ, Trắng, Đen.
User: Ok cám ơn bạn.
\end{lstlisting}

Trong cuộc hội thoại, khách hàng yêu cầu số lượng và màu sắc của sản phẩm với mã sản phẩm và kích thước mà họ cung cấp. Từ đó, ta sẽ có một mục tiêu người dùng như sau:

\renewcommand{\lstlistingname}{Ví dụ}
\begin{lstlisting}[caption={Một mục tiêu người dùng},label={exam:examgoal},language=code_vn,firstnumber=1]
{
    "request_slots": {
    	"amount_product": "UNK",
    	"color_product": "UNK"
    },
    "inform_slots": {
    	"name_product": "AT001",
    	"size_product": "XL"
    }
}
\end{lstlisting}

Khi quá trình huấn luyện diễn ra, tại mỗi cuộc hội thoại, bộ mô phỏng người dùng sẽ ngẫu nhiên chọn ra một mục tiêu người dùng từ danh sách và sẽ lần lượt gửi các hành động tương ứng phù hợp với mục tiêu người dùng đã được chọn tới tác nhân.

Để làm được điều trên, bộ mô phỏng người dùng cần phải lưu trữ trạng thái hội thoại cho riêng nó (trạng thái này khác với trạng thái của bộ quản lý trạng thái hội thoại). Trong đề tài này, nó sẽ được hiện thực theo cấu trúc dữ liệu dictionary trong Python, cụ thể:

\begin{itemize}
    \item \textbf{Rest slots:} Tất cả các \textit{inform slots} và \textit{request slots} từ mục tiêu người dùng mà tác nhân hoặc người dùng chưa được thông báo hoặc yêu cầu.
    \item \textbf{History slots:} Tất cả các cặp tên thông tin và giá trị từ các hành động của người dùng và tác nhân cho đến thời điểm hiện tại.
    \item \textbf{Request slots:} Các thông tin mà người dùng muốn yêu cầu tác nhân tại hành động hiện tại.
    \item \textbf{Inform slots:} Các thông tin sẽ được thông báo cho tác nhân tại hành động hiện tại.
    \item \textbf{Intent:} Ý định của hành động hiện tại.
\end{itemize}

Với mỗi hành động nhận được từ tác nhân, tùy vào \textit{intent} mà bộ mô phỏng người dùng sẽ phản hồi lại theo luật đã quy định sẵn:

Đầu tiên, kiểm tra số câu thoại đã thực hiện. Nếu số lần đến tối đa (20 lần) thì trả lời với "intent" = "done".

Nếu không thì tạo ra một hành động dựa trên \textit{intent} của tác nhân:

\begin{itemize}
    \item "request": tuỳ thuộc vào \textit{request slots}, phản hồi với "intent" = "inform" hoặc "request".
    \item "inform": tuỳ thuộc vào trạng thái hiện tại của cuộc hội thoại, phản hồi với "intent" = "inform" hoặc "request" hoặc "ok"
    \item "match\_found": tuỳ thuộc vào kết quả trả về của tác nhân, phản hồi với "intent" = "ok" hoặc "reject"
    \item "done": phản hồi với "intent" = "done", đồng thời tuỳ thuộc vào trạng thái hội thoại mà quyết định cuộc hội thoại có thành công hay không.
\end{itemize}

Một trong những thành phần quan trọng trong giải thuật nêu trên là \textit{reward} - phần thưởng cho mỗi hành động của tác nhân. Mỗi hành động phản hồi của bộ mô phỏng người dùng đều kèm theo điểm phần thưởng. Phần thưởng góp phần định hướng cho mỗi hành động của tác nhân: khuyến khích bằng cách cho điểm cao trước mỗi hành động chính xác hoặc dẫn đến thành công cho cuộc hội thoại, hoặc trừ điểm trước mỗi hành động không phù hợp hoặc sai lầm để tác nhân có thể tránh lặp lại trong tương lai. Tác nhân sẽ tự điều chỉnh hành vi sao cho tổng điểm thưởng nhận được là lớn nhất khi kết thúc hội thoại. Một số điểm thưởng đã được định nghĩa như sau:

\begin{itemize}
    \item \textbf{NO OUTCOME:} chưa thể kết thúc cuộc hội thoại. Đây là giá trị mặc định mà trải qua mỗi lượt trong cuộc hội thoại tác nhân sẽ bị trừ và điểm trừ này là thấp nhất. Nó có tác dụng kích thích tác nhân mau chóng tìm ra kết quả để kết thúc hội thoại sớm nhất có thể.
    \item \textbf{UNSUITABLE:} phản hồi không phù hợp. Đây là điểm trừ khi tác nhân yêu cầu thông tin mà người dùng đã yêu cầu hoặc thông báo từ trước, hoặc khi tác nhân yêu cầu lại thông tin đã yêu cầu trước đó.
    \item \textbf{FAIL:} kết thúc hội thoại nhưng thất bại vì không thoả mãn người dùng. Đây là điểm trừ lớn nhất.
    \item \textbf{GOOD INFORM:} tác nhân cung cấp giá trị hợp lý cho người dùng sẽ được cộng điểm thưởng khuyến khích hành vi này.
    \item \textbf{SUCCESS:} tác nhân phản hồi kết quả thoả mãn yêu cầu của người dùng khi kết thúc cuộc hội thoại. Điểm cộng này là lớn nhất.
\end{itemize}

\section{Bộ giả lập lỗi}
\label{sec:emc}
Sau khi nhận được hành động tạo ra từ bộ mô phỏng người dùng, nó sẽ được gửi đến bộ giả lập lỗi (EMC) để tạo ra các lỗi ngẫu nhiên. Việc này giúp tác nhân có thể xử lý được tốt hơn với các tình huống thực tế có thể phát sinh lỗi xuất phát từ các thành phần xử lý ngôn ngữ tự nhiên hoặc do người dùng mắc lỗi trong câu trả lời của họ. Các lỗi mà bộ phận này có thể tạo ra bao gồm lỗi ở thông tin của hành động thông báo và ý định của hành động. Cụ thể ở mức thông tin ta có ba lỗi với xác suất xuất hiện bằng nhau:

\begin{itemize}
    \item Thay thế giá trị bằng một giá trị ngẫu nhiên cho thông tin đó.
    \item Thay thế toàn bộ thông tin: chọn thông tin ngẫu nhiên và giá trị ngẫu nhiên cho thông tin đó.
    \item Xóa thông tin.
\end{itemize}

Đối với lỗi ở ý định, ta có thể tạo ra lỗi bằng cách thay thế ý định bằng một ý định ngẫu nhiên khác.
